\documentclass[letterpaper,12pt]{article}
\usepackage{tabularx} % extra features for tabular environment
\usepackage{amsmath}  % improve math presentation
\usepackage{amssymb}
\usepackage{dsfont}
\usepackage[table,xcdraw]{xcolor}
\usepackage[margin=1in,letterpaper]{geometry} % decreases margins
%\usepackage{graphicx} % Graphics library
\setlength\parskip{10pt} % New line after each paragraph
\setlength\parindent{0pt} % Remove indent
\usepackage[font={small}]{caption, subfig}
\renewcommand\refname{Bibliografía} % References title

\begin{document}

\title{Diseño de redes de transporte por Optimización de Colonia de Abejas}
\author{Martínez Flores Jorge Yael}
\date{30 de noviembre del 2018}
\maketitle

\section{Introducción}

Un correcto diseño de las redes de transporte puede mejorar el  modo en que 
opera el transporte público. Este corresponde a un problema de optimización 
de clase combinatorio, con una solución difícil de encontrar, ya que existen 
múltiples variables en juego, como el número de autobuses disponibles, la 
demanda de pasajeros y el presupuesto disponible y estas al tomar en cuenta 
diferentes intereses entran en conflicto. Entonces se trata de diseñar el 
problema como  un problema de decisión de múltiples criterios, buscando 
maximizar el número de pasajeros satisfechos y minimizar el número de autobuses 
y el tiempo de viaje de los pasajeros.

\section{Descripción del problema} 

Sea $G=(N,A)$ una red donde $N$ es el conjunto de nodos y $A$ el conjunto de 
aristas. Los nodos representan potenciales paradas de autobús mientras que 
las aristas representan las calles.

En el problema de diseño de redes de transporte buscamos el mejor conjunto de 
rutas $R$. Se buscan las aristas $A$ que se puedan incluir en $R$, de tal manera
que estas en conjunto formen las rutas de la red de tránsito.

El principal valor que se usa para describir el nivel de la red es el tiempo 
total de viaje tomado por los usuarios del servicio. Esta calidad de la 
solución generada es expresada en minutos. Se calcula el tiempo total de 
viaje de los pasajeros $T$ en la red de la siguiente manera:

$$ T = TT + w_1TTR + w_2 TU $$

\begin{itemize}
    \itemsep0em 
    \item $TT$: Tiempo a bordo de un vehículo de todos los pasajeros
    \item $TTR$: Número total de transbordos en la red
    \item $TU$: Número de pasajeros insatisfechos (cuando tiene que hacer dos o
    más transbordes)
    \item $w_1$: Penalización de tiempo por un transbordo
    \item $w_2$: Penalización de tiempo por un pasajero insatisfecho
\end{itemize}

La calidad toma en cuenta el número de transbordos, los pasajeros lo usan 
para cambiar de una ruta a otra, el número total de transbordos puede 
reducirse optimizando la red.

Sea $d_{ij}$ el número de viajes por unidad de tiempo entre el nodo $i$ y $j$.
Sea D la matriz origen-destino $O-D$ denotada como:

$$ D = \{ d_{ij} | i,j \in [1, 2, \dots , |N| ] \} $$

Sea $tr_{ij}$ el vehículo por el cuál se trasladan entre el nodo $i$ y el 
nodo $j$. Para este se tiene la matriz de tiempo de viaje:

$$ TR = \{ tr_{ij} | i,j \in [1, 2, \dots, |N|] \} $$

Suponiendo que nos dan la red $G=(N,A)$, la matriz $O-D$ y la de tiempo de 
viaje $TR$.Y que los pasajeros escogen la ruta más corta, podemos 
descomponer el diseño del transporte de redes en dos partes: La generación 
de las rutas de transporte y la determinación de la frecuencia del 
servicio para cada ruta.

En resumen: Dado un conjunto de $n$ nodos, la matriz $O-D$ que describe la 
demanda entre estos y la matriz de tiempo de viaje $TR$, genera un conjunto 
de rutas de tránsito en una red tal que exista una manera de minimizar 
el tiempo de viaje de los pasajeros $T$.

\section{Solución propuesta}

El problema de diseño de transporte en redes es  NP-Completo. Se ha intentado 
con diferentes heurísticas y aproximaciones para encontrar una solución a 
este problema. La propuesta para este caso está basada en la optimización de 
la colonia de abejas (BCO, Bee Colony Optimization). BCO representa una de 
las técnicas de inteligencia artificial de enjambre, la cuál es una técnica 
basada en el estudio de sistemas compuestos por individuos que se comunican 
entre ellos, intercambian información y realizan tareas en el mismo entorno.

Este algoritmo está inspirado en el comportamiento de las abejas en la 
naturaleza. Las abejas artificiales usan los principios del periodo de 
recolección de néctar. La idea principal en BCO es crear una colonia 
de abejas artificiales que sea capaz de resolver problemas de optimización 
combinatoria. Estas exploran a través del entorno, buscando soluciones 
factibles. Cada abeja artificial genera una solución al problema, pero para 
descubrir la solución más optima, las abejas colaboran e intercambian 
información, de esta manera se concentran en las áreas más prometedoras y van 
descartando las menos óptimas.

El algoritmo consiste en dos fases alternantes:
\begin{enumerate}
    \itemsep0em 
    \item \textit{Forward pass:} Cada abeja explora el entorno y en un número
    determinado de pasos construye o mejora la solución. Habiendo obtenido una 
    solución, las abejas regresan al nido y empieza la siguiente fase.
    \item \textit{Backward pass:} Todas las abejas comparten información de sus 
    resultados, comparten la calidad de su solución y cada abeja decide con una 
    cierta probabilidad abandonar la solución obtenida y volverse un seguidor 
    o bailar y así reclutar a compañeros antes de regresar a la solución 
    obtenida. Las abejas con soluciones de mejor calidad tienen una mayor 
    posibilidad de continuar su exploración. 
\end{enumerate}

Estas dos fases son realizadas iterativamente hasta que suceda una condición 
de paro como puede ser que el número máximo de fases haya terminado o que no 
haya incremento en la calidad de la solución, entre otras.

Se propone que el algoritmo BCO trabajando en diseño de redes de transporte 
primero generará una red de transporte inicial factible. Después, las abejas 
artificiales investigan el entorno a partir de la solución propuesta e 
intentan mejorarlo Suponemos que al inicio del diseño de la red, todas las 
abejas están en la colmena. La cual es una ubicación artificial que no está 
conectada con las líneas de la red.

\section{Solución inicial}

Consideremos la linea $l$ cuyos nodos terminales son $i$ y $j$.
Sea $N_l$ el conjunto de nodos conectados por la línea $l$. Esta linea 
será usada por los pasajeros que usarán el servicio directo (pasajeros que 
usan toda la línea desde el nodo de inicio hasta el nodo final), como habrá 
mas pasajeros que usarán la red con transbordo, denotamos a $ds_{ij}$ como 
el número de pasajeros que usarán el servicio directo:

$$ ds_{ij} = \sum_{m \in N_l} \sum_{n \in N_l}  d_{mn}$$

Denotamos a $DS$ como la matriz que contiene la información de los pasajeros que
usan este servicio directo:

$$ DS = \{ ds_{ij} | i,j \in [1, 2, \dots , |N|] \}$$

Se propone un algoritmo glotón para generar la solución inicial. En este se
tratará de conectar pares de nodos con los valores $ds_{ij}$ mas altos. De 
esta manera se intenta incrementar el número de pasajeros que usan el 
servicio directo.

\begin{enumerate}
    \itemsep0em 
    \item Sea $Y$ el número total de líneas ($NTL$) de autobuses en la red. 
    Sea $Y = \phi$. Sea $m = 1$.
    \item Sean $(a,b)$ el par de nodos con los valores $ds_{ij}$ más altos. 
    Ahora $a$ y $b$ son los nodos terminales de una nueva línea $l$ de autobuses.
    Se busca el camino más corto entre estos dos nodos. Los nodos que 
    pertenezcan a este camino serán las estaciones de línea de autobús. 
    Agrega $l$ a $Y$.
    \item Actualiza matriz $DS$ sin tener en cuenta la demanda de pasajeros 
    que se satisfacen.
    \item Si $m + NTL$, para. En otro caso, $m = m + 1$ y regresa al paso 2.
\end{enumerate}

\section{Modificación de la solución}

La modificación principal de BCO para este problema es el uso de  abejas 
heterogéneas. Es decir se resolverán el problema utilizando dos conjuntos 
diferentes de abejas artificiales. Abejas de tipo 1 y abejas de tipo 2. 
Estas difieren en la manera en la que modifican sus soluciones, respecto a 
hacer decisiones acerca de la lealtad, así como de unirse o ser reclutadora,
 ambas abejas se comportan de la misma manera.

\subsection{Abejas tipo 1}

La abeja tipo 1 escoge una línea del conjunto de líneas de autobuses basándose
en la probabilidad calculada de la siguiente manera:

$$ p_l = \frac{\frac{1}{ds_{ij}}}{\sum_{q \in L} \frac{1}{ds_{rs}}} $$

\begin{itemize}
    \itemsep0em 
    \item $i,j$: Terminales de la línea $l$
    \item $L$: Conjunto de líneas de autobuses
    \item $r,s$: Terminales de la línea $q$
    \item $ds_{ij}$: Número total de pasajeros que pueden viajar sin hacer algún
    transbordo.
\end{itemize}

Supongamos que la abeja tipo 1 escoge una línea cuyos terminales son $i$ y $j$.
La abeja 1 escoge entre uno de estos dos terminales. Supongamos que escoge $i$. 
La abeja tipo 1 descarta esta línea de autobús para crear una nueva línea, 
ahora entre el nodo $i$ y el nuevo nodo terminal $k$, el cuál puede ser 
escogido con la probabilidad mencionada anteriormente.

Suponemos que la abeja escoge el nodo $k$ para ser el nuevo terminal. La abeja 
busca el camino mas corto entre los nodos $i$ y $k$. Este camino mas corto es
la nueva ruta de autobús.

\subsection{Abejas tipo 2}

La abeja tipo 2 escoge una línea de autobús, después escoge de manera 
aleatoria entre los dos terminales de la línea. Supongamos escoge $j$. 
Ahora la abeja descarta este nodo terminal con probabilidad igual a $P$. 
La nueva línea generada de autobús contiene todas las anteriores paradas 
de autobús, excepto el terminal descartado. De esta manera esta línea es 
recortada. Si el nodo terminal sobrevive, se escoge una nueva parada de los 
nodos adyacentes al nodo terminal para ser agregada a la línea de autobús. 
En el caso para la gráfica, el nodo $k$ fue seleccionado para agregarse a la 
línea de autobús y la línea se expande.

\section{Reclutamiento}

En el caso en el que las abejas no quieran usar la solución generada 
anteriormente, la abeja comenzará a bailar y seguirá a otra abeja, las abejas 
mientras realizan su baile informan de diferentes soluciones a abejas cercanas. 
La probabilidad de que la solución del reclutador $b$ sea escogida por cualquier 
abeja es:

$$ p_b = \frac{O_b}{\sum^{R}_{k=1} O_k}, b = 1, 2, \dots, R $$
\begin{itemize}
    \itemsep0em 
    \item $O_k$: Función objetivo de la $k$-ésima solución
    \item $R$: Número de reclutadoras
\end{itemize}

Con esta probabilidad una abeja se unirá a una abeja que baila (o reclutadora). 
Las reclutadoras y sus compañeros reclutados realizan la exploración en el 
siguiente \textit{forward pass}.

\section{Evaluación}

Para un ejemplo de evalauación del algoritmo BCO se proponen los siguientes 
parámetros basados en la experiencia, esto valores tienden a que cada 
iteración sea más corta, pero no lo suficientemente larga como para obtener 
conocimiento de cada iteración.

\begin{itemize}
    \itemsep0em 
    \item Abejas: 10
    \item \textit{passes} por iteracion: 5
    \item Cambios en \textit{forward pass}: 2
    \item Penalización por pasajeros insatisfechos: 
    Tiempo promedio de viaje + 50 minutos
    \item Probabilidad de que una abeja destruya un terminal: 0.2
    \item Número de iteraciones: 400
\end{itemize}

Se usan los siguientes parámetros de comparación:

\begin{itemize}
    \itemsep0em 
    \item $d_0$: Porcentaje de demanda satisfecha sin transbordos
    \item $d_1$: Porcentaje de demanda satisfecha con un transbordo
    \item $d_2$: Porcentaje de demanda satisfecha con dos transbordos
    \item $d_{un}$: Porcentaje de demanda insatisfecha
    \item ATT: Promedio de tiempo de viaje.
\end{itemize}

El experimento es realizado en una red de autobús con 110 nodos y 275 aristas, 
el número total de viajes es 3,603,360. Se deben generar 55 líneas de autobús. 
El máximo de paradas por línea de autobús es 29.

En la tabla presentamos las dos soluciones propuestas, la primera por el 
algoritmo glotón y la segunda usando BCO, nótese que BCO mejora 
significativamente la solución inicial.

\begin{table}[tph!]
        \begin{tabular}{|l|l|l|}
        \hline
        \rowcolor[HTML]{DAE8FC} 
        Características \textbackslash Algoritmo & Algoritmo glotón              & BCO                  \\ \hline
        Viajes sin transbordo                    & $1,438,572$ ($d_0 = 39.42\%$) & ($d_0 = 59.65\%$)    \\ \hline
        Viajes con un transbordo                 & $1,607,740$ ($d_1 = 44.62\%$) & ($d_1 = 40.10\%$)    \\ \hline
        Viajes con dos transbordos               & $224,480$ ($d_2 = 6.23\%$)    & ($d_2 = 0.25\%$)     \\ \hline
        Pasajeros insatisfechos                  & $332,568$ ($d_{un} = 9.23\%$) & ($d_{un} = 0\%$)     \\ \hline
        Promedio de tiempo de viaje              & (ATT = $34.90$ min.)          & (ATT = $36.16$ min.) \\ \hline
        \end{tabular}
    \end{table}

El tiempo de viaje promedio ATT está compuesto de $36.16 = 34.13 + 20.3$

El primer sumando es el tiempo promedio de viaje, el segundo sumando es la 
penalización por usuario causado por él transborde. Para mejorar la estimación 
de la calidad calculamos el tiempo de viaje promedio por pasajero el cual 
usa el camino más corto en su viaje. Este tiempo promedio es $33.84$ min. 
Ninguna otra red de transporte debe ser menor a este valor, ya que representa 
la cota inferior. El tiempo de viaje promedio obtenido por BCO es 
relativamente cercano a la cota inferior, entonces podemos concluir que la 
calidad de la solución generada por BCO es relativamente alta.

\begin{thebibliography}{99}

\bibitem{Milos Nikolic, Dusan Teodorovic}
(2013) \textit{Transit network design by Bee Colony Optimization}. 
28 de noviembre del 2018, de ElSevier 
Sitio web: https://www.sciencedirect.com/science/article/pii/S0957417413002881

\end{thebibliography}

\end{document}
