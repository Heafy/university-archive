%++++++++++++++++++++++++++++++++++++++++
% Don't modify this section unless you know what you're doing!
\documentclass[letterpaper,12pt]{article}
\usepackage{tabularx} % extra features for tabular environment
\usepackage{amsmath}  % improve math presentation
\usepackage{amssymb}
\usepackage{dsfont}
\usepackage[margin=1in,letterpaper]{geometry} % decreases margins
\usepackage{cite} % takes care of citations
\usepackage[final]{hyperref} % adds hyper links inside the generated pdf file
\usepackage{graphicx} % Graphics library
\newcommand{\ecb}{elemento de combinación booleano }
\newcommand{\ccb}{circuito de combinación booleano }
\setlength\parskip{\baselineskip} % New line after each paragraph
\setlength\parindent{0pt} % Remove indent
\hypersetup{
	colorlinks=true,       % false: boxed links; true: colored links
	linkcolor=blue,        % color of internal links
	citecolor=blue,        % color of links to bibliography
	filecolor=magenta,     % color of file links
	urlcolor=blue         
}


\begin{document}

\title{Tarea 04}
\author{Martínez Flores Jorge Yael}
\date{\today}
\maketitle

\section{Knapsack Problem$^{1}$}
\subsection{Problema estandar}

Un item (o elemento) $i$ es un objeto con con valor $v_i$ y peso $w_i$ con $v_i, 
w_i \in \mathds{Z}^+$.

El problema knapsack (o problema de la mochila) consite en: Dado un conjunto
de $n$ elementos y un entero $K$.  Existe un límite $W$ del peso total de 
items que se pueden elegir. Se deben escoger algunos artículos 
(sin repeticiones) para maximizar el valor total.

Es decir, se busca el subconjunto 
$S \subseteq \{ 1, \dots , n \}$ tal que $\sum_{i \in S} w_i \le W$ y $\sum_{i \in S} 
v_i \ge K$

\subsection{PD. Knapsack es NP-Completo}

\subsubsection{PD. Knapsack $\in$ NP}

Un algoritmo determinista que corra en tiempo polinomial con una nd-choice que eliga
los items en la fase adivinadora mientras $w_i \le  W$. Posteriormente en la fase
verificadora se verifica si los items son los optimos para resolver el problema. Por 
lo tanto el problema Knapsack está en NP.

$$ \therefore \text{KNAPSACK } \in \text{NP} $$

\subsubsection{PD. Exact cover by 3-sets $\varpropto$ Knapsack}

Nos fijamos en el caso especial en el cual $v_i = w_i \ \forall i$ y $K = W$.
Entonces dado un conjunto de $n$ enteros $w_i, \dots, w_n$ y un entero $K$, 
se busca encontrar un subconjunto de enteros tal que la suma de los enteros dados 
sea exactamente $K$.

Reducimos el problema de la cobertura exacta de 3 vértices al problema de la 
mochila. Dado un ejemplar $\{ S_1, S_2, \dots, S_n \}$ de cobertura exacta de 3 
vértices, donde la pregunta es si existen conjuntos disjuntos entre estos que 
cubren el conjunto $U = \{1,2, \dots, 3m\}$.

Pensando en el conjunto dado como vectores en $\{  0,1 \}^{3m}$. Dichos vectores 
pueden ser vistos también como números binarios y la unión de los conjuntos se 
parece a la suma de enteros. La meta es encontrar un subjconjunto de dichos enteros 
tal que sumen $K = 2^n-1$. Hasta este punto la reducción parece completa.

Pero existe un "error" en esta reducción. La suma de números binarios es diferente 
a la unión de conjuntos en el sentido que esta lleva un acarreo por la suma de bits. 
Por ejemplo $3+5+7=15$ en binario daría como resultados 0011 + 0101 + 0111 = 1111. 
Pero sus conjuntos correspondientes $\{  3, 4 \}, \{ 2, 4 \}$ y $\{ 2, 3, 4 \}$ no 
son disjuntos, tampoco la unión $\{ 1, 2, 3, 4 \}$ lo es.

Existe una manera mas simple para resolver este problema: Considere los vectores 
como enteros que no están en base 2. Si no en base $n+1$. Esto es, el conjunto $S_i$ 
es el entero $w_i  = \sum_{j \in S_i} (n+1)^{3m-j}$.

Entonces ya no puede existir acarreo en la suma de los $n$ números. Podemos ver que 
hay un conjunto de estos números enteros que suman $K = \sum_{j = 0}^{3m-1} (n+1)^j$ 
si y solo si existe una cubierta de vértices exacta entre $\{  S_1, S_2, \dots, S_n \}$.

$$ \therefore \text{KNAPSACK } \in \text{NP-Completo} $$

\section{Circuit Sat$^{2}$}

Un \ecb es un circuito que tiene un número constante de entradas booleanas y salidas y realiza una función bien definida.
Los valores booleanos que puede representar son $\{ 0,1 \}$ donde $0$ significa
FALSE y 1 representa TRUE;

Un \ccb es uno o más elementos de combinación booleanos interconectados por cables.
Un cable puede conectarse de la salida de un elemento a la entrada de otro, de 
ese modo la salida del primer elemento es la entrada del segundo. El tamaño de un 
\ccb es el número de elementos de combinaciones booleanas mas el número de 
cables en el circuito.Un cable no puede tener mas de una salida de algún elemento 
conectado a este, pero puede tener varias entradas. Si no hay un elemento de 
salida conectado al cable, este es un circuito de entrada. Si no hay elementos 
de entrada conectado al cable, este es un circuito de salida. 

Una asignación de verdad para un \ccb es un conjunto de entradas booleanas. 
Decimos que un \ccb con una sola salida es satisfacible si tiene una asignación
de verdad que lo satisface. El problema de la satisfacción en circuitoS es: Dado un 
\ccb $C$ compuesto por compuertas lógicas $AND$, $OR$ y $NOT$, ¿$C$ es satisfacible?

Dado un circuito $C$, se debe intentar determinar si es satisfacible revisando
todas las posibles asignaciones a las entradas. Si el circuito tiene $k$ entradas y 
el tamaño de $C$ es polinomial en $k$, entonces se deben revisar $2^k$ posibles 
asignaciones.

\subsection{PD. Circuit Sat es NP-Completo}
\subsubsection{PD. Circuit-Sat $\in$ NP}

Se dan dos entradas a un algoritmo no determinista de tiempo polinomial $A$. La
primera entrada es un \ccb $C$, la otra entrada es un certificado que corresponde a
una asignación de valores booleanos a los cables de $C$.

Construimos $A$ de la siguiente manera: Para cada compuerta lógica en el circuito, 
revisa que el valor dado por el certificado en la salida del cable es correctamente
calculado como función de los valores en la entrada de los cables. Entonces, si la
salida del circuito es $1$, el algoritmo devuelve $1$, ya que los valores 
asignados a $C$ lo satisfacen. En otro caso $A$ devuelve $0$.

Siempre que un circuito satisfacible $C$ sea una entrada a un algoritmo $A$, existe
un certificado cuya longitud es es polinomial al tamaño de $C$ lo cual causa que la
salida de $A$ sea $1$. Siempre que un $A$ reciba como entrada a un circuito no 
satisfacible, no existirá un certificado para que el circuito sea satisfacible. 

$$ \therefore \text{CIRCUIT-SAT} \in \text{NP} $$

\subsubsection{PD. Circuit-Sat $\in$ NP-Completo}

Un programa es almacenado en la memoria de la computadora como una serie de 
instrucciones. Una instrucción contiene una operación a ser realizada, las 
direcciones de memoria que necesita y la dirección de memoria donde es almacenada.
Una dirección específica de memoria llamada contador del programa vigila
cual instrucción será ejecutada, aumentando automáticamente tras cada ejecución 
de una instrucción. 

Durante la ejecución de un programa, la memoria de la computadora almacena todo el
estado del calculo. Llamamos a cualquier estado particular de la memoria de la
computadora como una configuración, Entonces ahora la ejecución de una 
instrucción es el mapeo de una configuración a otra. El hardware que acompaña a este
mapeo puede ser representado como un \ccb.

Sea $L$ un lenguaje en NP. Debemos describir un algoritmo en tiempo polinomial $F$
para calcular una función de reducción $f$ que mapee cada cadena binaria $x$ a un 
circuito $C = f(x)$ tal que $x \in L$ si y solo si $C \in$ CIRCUIT-SAT.

Ya que $L \in NP$, debe existir un algoritmo $A$ que verifique a $L$ en tiempo 
polinomial. El algoritmo $F$ que se debe construir usa las dos entradas del 
algoritmo $A$ descrito anteriormente para calcular la función de reducción $f$.

Sea $T(n)$ el peor caso del algoritmo $A$ con una longitud de $n$ cadenas. Sea
$k \ge 1$ un número tal que $T(n) + O(n^k)$ con la longitud del certificado
$O(n^k)$ (El tiempo de ejecución de A es actualmente polinomial en el tamaño total
de la entrada, la cual incluye la cadena de entrada y el certificado).

La idea principal de la demostración es representar el cálculo de $A$ como una 
secuencia de configuraciones.El circuito combinado $M$, el cual implementado el 
hardware, mapea cada computadora $c_i$ a la siguiente configuración $c_{i+1}$, 
empezando desde la configuración inicial $c_0$. El algoritmo inicial $A$ 
devuelve la salida $0$ ó $1$ para algúna ubicación designada para el momento 
en el que termine su ejecución, si asumimos que a partir de entonces $A$ se 
detiene, entonces el valor nunca cambia. Esto si el algoritmo se ejecuta en 
el menos $T(n)$ pasos, la salida resulta como uno de los bits en $c_{T(n)}$.

El algoritmo de reducción $F$ construye un circuito  de combinación que calcula
todas las configuraciones producidas por una configuración inicial dada. La idea
es pegar $T(n)$ con copias del circuito $M$. La salida del $i$-ésimo circuito, el
cual produce la configuración $c_i$, es la entrada del $(i+1)$-ésimo circuito.
Entonces estas configuraciones, en lugar de ser almacenadas en la memoria, 
se quedan como valores en los cables conectados a las copias de $M$.

Dado una entrada $x$, debemos crear un circuito $C = f(x)$ que sea satisfacible si
y solo si existe un certificado $y$ tal que $A(x,y) = 1$. Cuando $F$ tiene la
entrada $x$, primero calcula $n = |x|$ y construye un \ccb $C'$ que contiene $T(n)$
copias de $M$. La entrada para $C'$ es una configuración inicial que corresponde a 
un cómputo en $A(x,y)$ y la salida es la configuración $c_{T(n)}$.

$F$ modifica el circuito $C'$ para construir el circuito $C=f(x)$ con los siguientes
 pasos:

\begin{enumerate}
	\item Conecta las entradas a $C'$ correspondiente al programa para $A$, al 
	contador inicial, la entrada $x$ y al estado inicial de la memoria directamente
	a estos valores. Por lo tanto las entradas restantes del circuito corresponden al
	certificado $y$.
	\item Ignora todas las salidas de $C'$, excepto por el bit $C(y) = A(x,y)$ 
	para cualquier entrada $y$ de longitud $O(n^k)$.
\end{enumerate}

El algoritmo de reducción $F$ cuando se le proporciona una cadena de entrada $x$, 
calcula dicho circuito $C$.

Necesitamos probar dos propiedades:

\begin{enumerate}
	\item Mostrar que $F$ calcula correctamente una función de reducción $f$. Esto es, 
	debemos mostrar que $C$ es satisfacible si y solo si existe un certificado $y$ 
	tal que $A(x,y) = 1$
	\item Mostrar que $F$ se ejecuta en tiempo polinomial. 
\end{enumerate}

supongamos que existe un certificado $y$ de longitud $O(n^k)$ tal que $A(x,y)=1$. 
Entonces, si aplicamos los bits de $y$ a las entradas de $C$, la salida de $C$ es 
$C(y)=A(x,y)=1$. Por lo tanto, si el certificado existe, entonces $C$ es 
satisfacible. Por lo tanto existe una entrada $y$ para $C$ tal que $C(y)=1$, por lo
que concluimos que $A(x,y)=1$. Por lo tanto $F$ calcula correctamente la función
de reducción.

Notese que el número de bits requeridos para representar una configuración es 
polinomial en $n$. El programa para $A$ tiene un tamaño constante, independiente
de la longitud de la longitud de su entrada $x$. La longitud de de la entrada $x$
es $n$ y la longitud del certificado $y$ es $O(n^k)$. Dado que el algoritmo se 
ejecuta en al menos $O(n^k)$ pasos, la cantidad de memoria para el trabajo requerido
por $A$ es polinomial en $n$.

El circuito $C$ consiste en a lo más $t=O(n^k)$ copias de $M$ y por lo tanto
tiene tamaño polinomial en $n$. El algoritmo de reducción $F$ puede construir $C$
a partir de $x$ en tiempo polinomial, dado que cada paso de la construcción tiene
tiempo polinomial.

$$ \therefore \text{CIRCUIT-SAT} \in \text{NP-Completo} $$

\begin{thebibliography}{99}

\bibitem{Papadimitriou}
[201 -203] Christos H. Papadimitriou, \textit{Computational Complexity},
(University of California, San Diego, 1995).

\bibitem{Cormen}
[987 - 994] Thomas H. Cormen, Charles E. Leiserson, Ronald L. Rivest and 
Clifford Stein, \textit{Introduction to Algorithms},
(MIT Press, Massachusetts, 2009).

\end{thebibliography}

\end{document}
