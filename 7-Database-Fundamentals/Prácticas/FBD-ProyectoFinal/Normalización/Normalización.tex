\documentclass{article}
\usepackage[T1]{fontenc} % Use 8-bit encoding that has 256 glyphs
\usepackage[spanish]{babel} % Spanish language
\selectlanguage{spanish}
\usepackage[utf8]{inputenc} % For accents
\usepackage{geometry}
\usepackage{textcomp}
\usepackage{amssymb}
\geometry{a4paper, total={150mm,240mm}, left=25mm, top=25mm} 
\newcommand{\flecha}{$\,\to\,$ }
\title{Normalización}
\author{Computólogos A.C. \and Taquerías Tacoste}
\date{Ultima actualización: Enero 05, 2018}

\begin{document}

\maketitle

Usando la Forma Normal de Boyce-Codd (BCNF), utilizando las Dependencias Funcionales anteriormente planteadas en el archivo correspondiente procedemos a normalizar:

\begin{itemize}

\item \texttt{TaqueroMucho(\ldots)}
	\begin{itemize}
		\item \textbf{CURP, idTarjeta \flecha CantidadPuntos.}\\
		Como CURP, idTarjeta es una super llave, es trivial.
	\end{itemize}

\item \texttt{Cliente(\ldots)}
	\begin{itemize}
		\item \textbf{CURP \flecha Nombre(s), Ap\_Paterno, Ap\_Materno.}\\
		Es decir, cada cliente tiene solamente un nombre.		
		\item \textbf{CURP \flecha Calle, Colonia, CP, Estado, NumExt, NumInt}\\
		Ya que cada cliente tiene solamente una dirección.
		\item \textbf{CURP \flecha Telefono.}\\
		Pues cada cliente tiene solamente un teléfono.
		\item \textbf{CURP \flecha Email.}\\
		Aquí, con cada cliente tiene solamente un correo.
		\item \textbf{CURP \flecha Sexo.}\\
		En cada cliente solamente tiene un sexo.
		\item \textbf{CURP \flecha Nombre(s), Ap\_Paterno, Ap\_Materno, Calle, Colonia, CP, Estado, NumExt, NumInt, Telefono, Email, Sexo.}\\
		Usando unión, tenemos que CURP es super llave.	
	\end{itemize}

\item \texttt{Comprar(\ldots)}
	\begin{itemize}
		\item \textbf{$\hbox{CURP} \twoheadrightarrow \hbox{Tipo.}$}\\
		Como CURP, Tipo es una super llave, es trivial.
	\end{itemize}

\item \texttt{Salsa(\ldots)}
	\begin{itemize}
		\item \textbf{Tipo \flecha NivelPicor, Recomendaciones.}\\
		Como Tipo es una super llave, es trivial.
	\end{itemize}

\item \texttt{Consumir(\ldots)}
	\begin{itemize}
		\item \textbf{CURP, idConsumible, Fecha \flecha ADomicilio.}\\
		Como CURP, idConsumible, Fecha es una super llave, es trivial.
	\end{itemize}

\item \texttt{Consumible(\ldots)}
	\begin{itemize}
		\item \textbf{idConsumible \flecha Nombre, Cantidad, Descripción, Precio.}\\
		Como idConsumible es una super llave, es trivial.
	\end{itemize}

\item \texttt{Entradas(\ldots)}
	\begin{itemize}
		\item \textbf{idConsumible \flecha idConsumible.}\\
		Es trivial.
	\end{itemize}

\item \texttt{DelCazo(\ldots)}
	\begin{itemize}
		\item \textbf{idConsumible \flecha idConsumible.}\\
		Es trivial.
	\end{itemize}

\item \texttt{Sopes/Huaraches(\ldots)}
	\begin{itemize}
		\item \textbf{idConsumible \flecha idConsumible.}\\
		Es trivial.
	\end{itemize}

\item \texttt{Enchiladas(\ldots)}
	\begin{itemize}
		\item \textbf{idConsumible \flecha idConsumible.}\\
		Es trivial.
	\end{itemize}

\item \texttt{Quesos(\ldots)}
	\begin{itemize}
		\item \textbf{idConsumible \flecha idConsumible.}\\
		Es trivial.
	\end{itemize}

\item \texttt{Ensaladas(\ldots)}
	\begin{itemize}
		\item \textbf{idConsumible \flecha idConsumible.}\\
		Es trivial.
	\end{itemize}

\item \texttt{Gringas/Quecas/Volcanes(\ldots)}
	\begin{itemize}
		\item \textbf{idConsumible \flecha idConsumible.}\\
		Es trivial.
	\end{itemize}

\item \texttt{Alambres(\ldots)}
	\begin{itemize}
		\item \textbf{idConsumible \flecha idConsumible.}\\
		Es trivial.
	\end{itemize}

\item \texttt{Hamburguesas(\ldots)}
	\begin{itemize}
		\item \textbf{idConsumible \flecha idConsumible.}\\
		Es trivial.
	\end{itemize}

\item \texttt{Tacos(\ldots)}
	\begin{itemize}
		\item \textbf{idConsumible \flecha idConsumible.}\\
		Es trivial.
	\end{itemize}

\item \texttt{Tortas(\ldots)}
	\begin{itemize}
		\item \textbf{idConsumible \flecha idConsumible.}\\
		Es trivial.
	\end{itemize}

\item \texttt{Bebidas(\ldots)}
	\begin{itemize}
		\item \textbf{idConsumible \flecha idConsumible.}\\
		Es trivial.
	\end{itemize}

\item \texttt{Postres(\ldots)}
	\begin{itemize}
		\item \textbf{idConsumible \flecha idConsumible.}\\
		Es trivial.
	\end{itemize}

\item \texttt{Empleado(\ldots)}
	\begin{itemize}
		\item \textbf{CURP \flecha Nombre(s), Ap\_Paterno, Ap\_Materno.}\\
		 Es decir, cada empleado tiene solamente un nombre.		
		\item \textbf{CURP \flecha Calle, Colonia, CP, Estado, NumExt, NumInt}\\
		Ya que cada empleado tiene solamente una dirección.
		\item \textbf{CURP \flecha Telefono.}\\
		Pues cada empleado tiene solamente un teléfono.
		\item \textbf{CURP \flecha Email.}\\
		Aquí, con cada empleado tiene solamente un correo.
		\item \textbf{CURP \flecha Sexo.}\\
		Cada empleado tiene solamente un sexo.
		\item \textbf{CURP \flecha Nombre(s), Ap\_Paterno, Ap\_Materno, Calle, Colonia, CP, Estado, NumExt, NumInt, Telefono, Email, Sexo.}\\
		Usando unión, tenemos que CURP es super llave.	
	\end{itemize}

\item \texttt{Parrilero(\ldots)}
	\begin{itemize}
		\item \textbf{CURP \flecha CURP.}\\
		Es trivial.
	\end{itemize}

\item \texttt{Taquero(\ldots)}
	\begin{itemize}
		\item \textbf{CURP \flecha CURP.}\\
		Es trivial.
	\end{itemize}

\item \texttt{Mesero(\ldots)}
	\begin{itemize}
		\item \textbf{CURP \flecha CURP.}\\
		Es trivial.
	\end{itemize}

\item \texttt{Cajero(\ldots)}
	\begin{itemize}
		\item \textbf{CURP \flecha CURP.}\\
		Es trivial.
	\end{itemize}

\item \texttt{Tortillero(\ldots)}
	\begin{itemize}
		\item \textbf{CURP \flecha CURP.}\\
		Es trivial.
	\end{itemize}

\item \texttt{Repartidor(\ldots)}
	\begin{itemize}
		\item \textbf{CURP \flecha NumeroLicencia, Transporte.}\\
		Es trivial, ya que repartidor es una super llave.
	\end{itemize}

\item \texttt{Trabajar(\ldots)}
	\begin{itemize}
		\item \textbf{CURP \flecha ID\_Sucursal.}\\
		Un empleado solamente puede trabajar en una sucursal.
		\item \textbf{$\hbox{ID\_Sucursal} \twoheadrightarrow \hbox{CURP.}$}\\
		Una sucursal puede tener muchos empleados.
	\end{itemize}

\item \texttt{Sucursal(\ldots)}
	\begin{itemize}
		\item \textbf{ID\_Sucursal \flecha Nombre.}\\
		Como ID\_Sucursal es una super llave, es trivial.
	\end{itemize}

\item \texttt{Tener(\ldots)}
	\begin{itemize}
		\item \textbf{$\hbox{ID\_Sucursal} \twoheadrightarrow \hbox{idProducto.}$}\\
		Una sucursal puede tener muchos productos.
		\item \textbf{$\hbox{idProducto} \twoheadrightarrow \hbox{idSucursal.}$}\\
		Un producto puede estar en muchas sucursales.
	\end{itemize}	

\item \texttt{Producto(\ldots)}
	\begin{itemize}
		\item \textbf{idProducto \flecha FechaCompra, FechaCaducidad, UnidadesInventario, Marca, Precio, Proveedor.}\\
		Es decir, cada producto tiene solamente una fecha de compra, fecha de caducidad, unidades en inventario, marca y precio.\\
		Como idProducto es una super llave, es trivial.
	\end{itemize}

\end{itemize}

\end{document}
\grid
