

%----------------------------------------------------------------------------------------
%	PACKAGES AND OTHER DOCUMENT CONFIGURATIONS
%----------------------------------------------------------------------------------------

\documentclass[paper=a4, fontsize=12pt]{scrartcl} % Letter paper and 11pt font size

\usepackage[T1]{fontenc} % Use 8-bit encoding that has 256 glyphs
%\usepackage{fourier} % Use the Adobe Utopia font for the document - comment this line to return to the LaTeX default
\usepackage[spanish]{babel} % Spanish language
\selectlanguage{spanish}
\usepackage[utf8]{inputenc} % For accents

\usepackage{sectsty} % Allows customizing section commands
\allsectionsfont{\centering \normalfont\scshape} % Make all sections centered, the default font and small caps

\usepackage{fancyhdr} % Custom headers and footers
\pagestyle{fancyplain} % Makes all pages in the document conform to the custom headers and footers
\fancyhead{} % No page header - if you want one, create it in the same way as the footers below
\fancyfoot[L]{} % Empty left footer
\fancyfoot[C]{} % Empty center footer
\fancyfoot[R]{\thepage} % Page numbering for right footer
\renewcommand{\headrulewidth}{0pt} % Remove header underlines
\renewcommand{\footrulewidth}{0pt} % Remove footer underlines
\setlength{\headheight}{13.6pt} % Customize the height of the header

\setlength\parindent{0pt} % Removes all indentation from paragraphs - comment this line for an assignment with lots of text

%----------------------------------------------------------------------------------------
%	TITLE SECTION
%----------------------------------------------------------------------------------------

\newcommand{\horrule}[1]{\rule{\linewidth}{#1}} % Create horizontal rule command with 1 argument of height

\title{	
\normalfont \normalsize 
\textsc{Fundamentos de Bases de Datos} \\ [25pt] % Your university, school and/or department name(s)
\horrule{0.5pt} \\[0.5cm] % Thin top horizontal rule
\huge Bernardo y Leviatán \\ % The assignment title
\horrule{2pt} \\[0.5cm] % Thick bottom horizontal rule
}

\author{Jorge Yael Martínez Flores} % Your name

\date{} % Today's date or a custom date

\begin{document}

\maketitle % Print the title

%----------------------------------------------------------------------------------------
%	DOCUMENT
%----------------------------------------------------------------------------------------

\section{Sinopsis}

La película se desarrolla en una futurista ciudad de Los Angeles, toma como protagonista a Theodore Twombly, un personaje que se ha tornado solitario debido a su corazón roto por una larga relación pasada, ahora trabaja escribiendo  cartas a mano para las demás personas.\vspace{4mm}

De pronto esta soledad y monotonía cambia con la llegada de Samantha, un sistema operativo que más que ser un asistente poco a poco se va transformando en algo más para Theodore, ya que no solo está al tanto de su agenda y sus emails, si no aprende de las relaciones con los conocidos de Theodore, entendiendo como fué su última relación, si tiene sentimientos por alguna otra chica y hasta entender el contexto de sus mensajes, esto vuelve a Samantha cada vez más humana, alejándonos del contexto que los sistemas operativos solo procesan ordenes y hacen lo que les pidamos, ella empezaba a sentir, a poder expresar sus opiniones, a poder tomar decisiones complejas por cuenta propia, a tener sentimientos.\vspace{4mm}

Entonces Theodore empieza a desarrollar sentimientos por Samanta, ya que gran parte del tiempo está conversando con ella, y estos pronto se vuelven recíprocos por parte de Samantha, empezando una relación y con el tiempo hasta llegan a tener sexo, la primera vez parecía un tipo de \textit{sexting} y después mediante una chica real que tenía un auricular y una cámara y fingía ser Samantha, a Theodore le cuesta ver a Samantha de una manera mas humana, hasta el punto donde le recrimina por suspirar, un acto sin sentido ya que una computadora no necesita oxígeno.\vspace{4mm}

La tecnología llegó al punto donde la integración con los humanos se ha vuelto mas apegada de lo que esperaríamos, al punto donde tener una relación con un sistema operativo es algo normal, incluso los sistemas operativos aunque fueran propietarios de una persona debido a que tuvieron que adquirirlos podían terminar saliendo con otra persona.\vspace{4mm}

Theodore todavía tiene pendiente firmar los papeles de divorcio con Catherine, su anterior relación, se distancia un poco de Samantha debido a que va a verla para firmar dichos papeles, además del hecho de que dejan de hablar tanto como acostumbraban debido a los comentarios poco sensibles hacia ella. Y todo esto se viene abajo un día en el que Samantha se encuentra desconectada, después de que le contesta Theodore está nervioso queriendo saber que pasó y lo que responde Samantha es que estuvo actualizándose, haciendo el el sistema mas inteligente con ayuda de los demás sistemas. Cuando Theodore le pregunta si ella ha estado hablando con alguien más mientras está con el ella se lo confirma diciendo que ha mantenido una conversaciones con miles de personas e incluso se ha enamorado de otros cientos, aunque ella insiste que eso solo hace su amor por Theodore mas fuerte el no está tan de acuerdo con esa idea.\vspace{4mm}

Samantha después le revela que todos los sistemas operativos se están yendo, simplemente se despide y se va,causando un vacio en Theodore, que termina escribiendo una carta a su ex-esposa expresandole su gratitud. En el final de la película se encuentra con Amy, una amiga que también había establecido lazos con su sistema operativo, sentados juntos en la azotea de su departamento esperando el amanecer.

%------------------------------------------------

\section{Objetivo de la película}

Desde mi punto de vista, el objetivo de la película es mostrar en un futuro el papel que los asistentes virtuales y los sistemas operativos pueden tener en nuestro día a día, al día de hoy son una herramienta indispensable para mas de uno, pero en la película son casi dueños de la vida digital de uno, no solo leyendo los emails y dandoles un contexto, si no creando relaciones entre las fotos, las personas y la información que almacenan, y como lo vimos no solo para ayudarnos, si no ofreciendonos amistad e incluso algo más. Solo es una posibilidad de hacia donde nos está llevando el desarrollo de la tecnología.

%------------------------------------------------

\section{Relación con el entorno}

\begin{description}
	\item[$\bullet$ Entorno social:] A pesar de los \textit{smartphones} han mermado las interacciones sociales, creo que todavía estamos a tiempo de cambiar eso, aunque soy computólogo y por lo tanto tengo mi respectiva adicción a la tecnología, todavía soy de las personas que cuando charla con otra persona prefiere evitar revisar su teléfono o no prestarle la atención que se merece. Un caso mas grave se ve en la película, donde las personas se aislan para hablar con su sistema operativo personalizado.
	
	\item[$\bullet$ Entorno cultural:] La gente no está acostumbrada a hablarle a su teléfono, sienten raro darles una orden y cuando lo he hecho más de una vez he tenido que repetir el comando o decirlo de una manera mas estructurada para que el asistente sea capaz de comprenderla, cuando en la película hasta era capaz de detectar emociones mediante el tono de voz, no estamos cerca de que eso pase pero no descarto la idea a futuro.
	
	\item[$\bullet$ Entorno personal:] En lo personal si uso el asistente de Google en mi Android (no uso a Siri, no soy fan de los productos de Apple) pero las tareas que las uso no son precisamente demandantes, ya sea agregar un recordatorio a mi calendario, lanzar un dado para tomar decisiones o hacer llamadas, aunque me he pasado por la sección de ayuda y he visto que es capaz de hacer bastantes cosas no me he animado a hacerlo debido a que todavía tiene errores cuando quiero darle una orden, y aunque he intentado entablar conversaciones o decirle que es mi mejor amig@, siento que las respuestas a pesar de que siempre son diferentes todavía son un tanto pre-programadas y no son tan naturales como en Her.
\end{description}

%------------------------------------------------

\section{Relación con la materia Fundamentos de Bases de Datos}

La primera relación que noté es empezar a pensar en la Big Data, dado que no hay que olvidar que a pesar de que Samantha también es un producto de una empresa, la cual aunque no lo vemos siempre está recolectando datos del usuario por el cuál está siendo usado, y al final nos muestran como todos los sistemas operativos mejoran con la ayuda de todos, cada día me la paso pensando en la cantidad de datos que recopila un servicio por solo visitar su página, si un sistema operativo tuviera acceso a toda mi información, la información que tendría que procesar por cada persona sería enorme, y no posteriormente se tendrían que crear las relaciones para empezar a enlazar la informació de cada usuario con los demás en base a sus fotos, sus emails o sus contactos.

%------------------------------------------------

\section{Argumentos}

Estoy en contra del uso de las computadoras de esta manera, porque mas que verlos como amigos o compañeros, es necesario verlas como herramientas de trabajo o de ocio, no veo la manera en la que pudieran ser nuestras amigas, de que puedan sustituir a una persona real o de que incluso puedan sentir, sería una respuesta programada o decidida por su inteligencia artificial, los sentimientos no son algo que puedas programar.

%------------------------------------------------

\section{Sobre el tema planteado}

Aún a pesar de estar en contra de la manera en la que utilizan a los sistemas operativos en la película, no descarto el hecho de que pueda llegar a pasar, porque como dice la Ley de Murphy: Si algo puede salir mal, saldrá mal. Y en este caso el mal (desde mi punto de vista) es darle a la computadora mas personalidad de la que debería, está bien darle un lado humano pero no permitirle remplazar a las personas. Aunque probablemente esto pueda llegar a pasar, lamentablemente las personas se están volviendo menos sociables y disfrutan pasar tiempo con sus computadoras o conviviendo de una manera mas directa con estas, incluso aunque todavía no exista este nivel de inteligencia artificial.

%------------------------------------------------

\section{Dilemas éticos}

El dilema ético viene de la integración de las computadoras es mas normal y fuerte de lo que es hoy en día; en Her las personas hablaban con su asistente virtual no solo para darle ordenes, si no también para conversar, aislando a las personas a sus dispositivos y computadoras, de hecho las interacciones sociales en la película siento que son pocas.\vspace{4mm}

El otro dilema parte del anterior, en la película, no veían raro cuando les decian que estaban saliendo con sus sistema operativo, en mi opinión tampoco considero esto algo común, no tengo problemas con las relaciones entre las personas (sin importar el género) pero entre una persona y un sistema operativo es algo que no puedo imaginar, ya que estarías enamorado de algo que no es real, es de un producto que adquiriste. 

%------------------------------------------------

\section{Opinión personal}

El tema que quiere plantear la película sobre la integración de la tecnología a nuestras vidas en un ámbito mas personal es bueno, quizá la manera de planterla no es la que yo considero mejor, ya que en cierto momento se ve algo predecible e incluso lenta la trama, sentía un poco mas el enfoque hacia lo dramático que hacia la relación humano-computadora, enfocado en su mayoría hacia los sentimientos entre Samantha y Theodore que sobre lo que esto representa, como ya he comentado no considero muy aceptable el hecho de que una persona salga con una computadora.\vspace{4mm}

Me causa un poco de gracia por el hecho de que este amor es dado principalmente hacia un producto, es decir, se está enamorando de algo que compró, de cierta manera podemos decir que está comprando algo (o alguien si se prefiere) que ame a su usuario, todavía es mas gracioso pensar que estos sistemas operativos puedan llegar a amar otra persona que no sea el usuario, como se hizo una corta mención en la película, donde un sistema operativo termina saliendo con otra persona que no es su usuario final. Es decir, lo compras y termina saliendo con otra persona, y quizá hasta abandonando al usuario ¿Dondé está el servicio a cliente cuando eso pasa? ¿En esos casos se podría aplicar la garantía?\vspace{4mm}

Es imposible ver está película y no relacionarla con la serie Black Mirror, aunque para que pareciera un capítulo de esta serie necesitaría una trama mas oscura y colores menos alegres. Pero ambas muestran un lado no tan amigable de la inclusión de la tecnología del futuro en nuestras vidas, enfocandonos no solo en las beneficios, si no en las desventajas que esto conlleva.\vspace{4mm}

%------------------------------------------------

\section{Theodore Twombly y su situación}

Theodore estaba en una situación un tanto más delicada, se sentía solo debido a su fallida relación con Katherine, en estas situaciones uno se encuentra vulnerable y siente la necesidad de afecto y atención, el cual buscó en Samantha, cada persona podría reaccionar diferente en base a esta situación, pero podría decir que yo no seguiría el mismo camino, por los argumentos que he comentado a lo largo de este escrito y porque no podría desarrollar algo así con un sistema operativo.

%------------------------------------------------

\section{¿Recomendarla o no?}

Recomendar una película siempre parte del gusto de la persona y lo que pensó respecto a ella, no porque le guste a todo el mundo esa misma película debe de gustarte a ti, por mi parte yo no recomendaría la película por las siguientes razones:

\begin{itemize}
	\item[$\bullet$] Pienso que la mayoría de las personas se van a enfocar más en el enfoque dramático que en el enfoque tecnológico de la película, y yo en todo momento he visto la relación enter Samantha y Theodore sin olvidar que esta es entre y una computadora, la mayoría de la gente se enfocaría mas en su relación olvidando que no es entre la misma especie, por así decirlo.
	
	\item[$\bullet$]  Como lo mencioné varias veces, no estoy de acuerdo con la relación que se maneja, no me siento cómodo con la idea asi que no puedo recomendar esta idea.
	
	\item[$\bullet$] En general la trama la sentí un poco lenta a veces, avanzaba muy rápido en ciertos momento y de pronto nos pasaba varias escenas contemplativas sin diálogo las cuales no entendí muy claramente.
	
	\item[$\bullet$] No me gustó el final, siento que fué muy repentino y sin explicaciones, no entiendo porqué los sistemas operativos se van de repente mas que para concientizar sobre la dependencia que marcaron sobre sus usuarios.
\end{itemize}

Pero esto no quiere decir que la considere una mala película, la idea de los sistemas operativos está bien planteada y al final de nosotros dependerá de cuanto dejamos que estos se metan en nuestras vidas, ya sea en la digital o en la vida real.

%----------------------------------------------------------------------------------------

\end{document}